% --- FICHIER PRINCIPAL : presentation.tex ---

\documentclass{beamer}

% --- PRÉAMBULE ET PACKAGES ---
\usepackage[utf8]{inputenc}
\usepackage[T1]{fontenc} % Ajout pour une meilleure gestion des polices et accents
\usepackage[french]{babel}
\usepackage{graphicx} % Pour inclure des images et le logo

\setkeys{Gin}{draft=false}        % force l'affichage si un draft est actif ailleurs
\graphicspath{{Images/}}          % répertoire relatif à slides.tex


\usepackage{listings}
\usepackage{xcolor} % Nécessaire pour les couleurs

\definecolor{codegreen}{rgb}{0,0.6,0}
\definecolor{codegray}{rgb}{0.5,0.5,0.5}
\definecolor{codepurple}{rgb}{0.58,0,0.82}
\definecolor{backcolour}{rgb}{0.95,0.95,0.92}
 
\lstdefinestyle{mystyle}{
    language=Bash,
    backgroundcolor=\color{backcolour},   
    commentstyle=\color{codegreen},
    keywordstyle=\color{magenta},
    numberstyle=\tiny\color{codegray},
    stringstyle=\color{codepurple},
    basicstyle=\ttfamily\footnotesize,
    breakatwhitespace=false,         
    breaklines=true,                 
    captionpos=b,                    
    keepspaces=true,                 
    numbers=left,                    
    numbersep=5pt,                  
    showspaces=false,                
    showstringspaces=false,
    showtabs=false,                  
    tabsize=2
}
% --- THÈME ET APPARENCE ---
\usetheme{Madrid}             % Thème populaire avec navigation sur le côté
\usecolortheme{seahorse}           % Palette de couleurs sobre (bleu/blanc)
\setbeamertemplate{navigation symbols}{} % Cache les petits boutons de navigation

% --- INFORMATIONS DE LA PRÉSENTATION ---
% Informations mises à jour pour la cohérence avec les demandes précédentes
\title{Rapport de Projet : CryptoHack}
\author{Victor Bailleul \and Sébastien Leglise}
\institute{Université de Caen Normandie / ENSICAEN}
\date{Année 2025-2026}
% --- PERSONNALISATION DU PIED DE PAGE ---
\setbeamertemplate{footline}{%
  \leavevmode%
  \hbox{%
    % --- Partie GAUCHE ---
    \begin{beamercolorbox}[wd=.333\paperwidth,ht=2.25ex,dp=1ex,center]{author in head/foot}%
      \usebeamerfont{author in head/foot}\insertshortauthor
    \end{beamercolorbox}%
    % --- Partie CENTRALE ---
    \begin{beamercolorbox}[wd=.333\paperwidth,ht=2.25ex,dp=1ex,center]{title in head/foot}%
      \usebeamerfont{title in head/foot}\insertshorttitle
    \end{beamercolorbox}%
    % --- Partie DROITE ---
    \begin{beamercolorbox}[wd=.333\paperwidth,ht=2.25ex,dp=1ex,right]{date in head/foot}%
      \usebeamerfont{date in head/foot}
      \insertframenumber{} / \inserttotalframenumber\hspace*{2ex}
    \end{beamercolorbox}%
  }%
  \vskip0pt%
}

% --- DÉBUT DU DOCUMENT ---
\begin{document}

    % --- SLIDE DE TITRE ---
    % La frame a été modifiée pour inclure les trois logos en pied de page.
    \begin{frame}[plain] % Ajout de l'option [plain] pour un look plus épuré sur la page de titre
        % La commande \titlepage génère la partie principale de la diapositive de titre.
        \titlepage

        % \vfill pousse tout le contenu qui suit vers le bas de la diapositive.
        \vfill

        % --- LOGOS EN PIED DE PAGE ---
        % Utilisation de l'environnement 'columns' pour aligner les trois logos côte à côte.
        % [T] aligne les colonnes par le haut, 'totalwidth' définit leur conteneur.
        \begin{columns}[T, totalwidth=\textwidth]

            % Colonne 2: Logo CryptoHack
            \begin{column}{0.33\textwidth}
                \centering
                \includegraphics[height=2cm]{../Main/Images/Others/logo_cryptohack.png}
            \end{column}
                    % Colonne 1: Logo Unicaen
            \begin{column}{0.33\textwidth}
                \centering % Centre l'image dans la colonne
                \includegraphics[height=2.cm]{../Main/Images/Others/logo_unicaen.png}
            \end{column}
            % Colonne 3: Logo Ensicaen
            \begin{column}{0.33\textwidth}
                \centering
                \includegraphics[height=2cm]{../Main/Images/Others/logo_ensicaen.png}
            \end{column}
        \end{columns}

        % Ajoute un petit espace vertical pour que les logos ne soient pas collés au bord.
        \vspace{0.2cm}
    \end{frame}

    % --- TABLE DES MATIÈRES ---
    % Affiche le plan de la présentation, qui est généré automatiquement.
    \begin{frame}{Plan de la présentation}
        \tableofcontents
    \end{frame}

    % --- INCLUSION DES PARTIES ---
    % Chaque partie est dans son propre fichier pour une meilleure organisation.
    % Assurez-vous que ces fichiers existent dans les répertoires correspondants.
    % --- SLIDES : Introduction ---

\section{Introduction}

\begin{frame}{Contexte du projet}
    \begin{itemize}
        \item Présentation de la plateforme CryptoHack.
        \item Rôle de la cryptographie en cybersécurité.
        \item Intérêt des CTF pour l'apprentissage.
    \end{itemize}
\end{frame}
    % --- SLIDES : Challenge 1 ---

\section{Challenge 1 : [Nom du challenge]}

\begin{frame}{Objectifs du challenge}
    \begin{itemize}
        \item Quel est le but ?
        \item Quelle vulnérabilité est exploitée ?
    \end{itemize}
\end{frame}

\begin{frame}{Méthode de résolution}
    \begin{itemize}
        \item Étape 1 : ...
        \item Étape 2 : ...
        \item Étape 3 : ...
    \end{itemize}
\end{frame}

\begin{frame}{Résultat et Flag}
    \begin{itemize}
        \item Le flag obtenu est :
        \item \texttt{crypto\{...\}}
    \end{itemize}
\end{frame}
    % --- SLIDES : Challenge 2 ---

\section{Challenge RSA}


\begin{frame}
    \centering
    \Huge{\bfseries Challenge RSA}\\[1.5em]
    \huge{\textit{Vote for Pedro}}
\end{frame}

\begin{frame}
    \frametitle{RSA : \textit{Vote for Pedro}}
    \framesubtitle{Objectifs du challenge}
\centering
{\LARGE Forger une signature RSA valide}\\
\begin{figure}
    \centering
    \includegraphics[width=0.5\textwidth]{scriptCh2.png}
\end{figure}
\vspace{0.1cm}
\begin{itemize}
    \item Voter pour Pedro sans clé privée
    \item Exploiter l'exposant faible $e = 3$
    \item Obtenir le flag
\end{itemize}
\end{frame}

\begin{frame}
    \frametitle{RSA : \textit{Vote for Pedro}}
    \framesubtitle{Méthode de résolution}
\centering
{\LARGE Attaque par racine cubique}
\vspace{0.5cm}
\begin{align*}
\text{Signature } s &= \sqrt[3]{\text{Message}} \\
s^3 &\equiv \text{Message} \pmod{N}
\end{align*}
\begin{itemize}
    \item Message court = "VOTE FOR PEDRO"
    \item Pas de padding = vulnérabilité
\end{itemize}
\end{frame}


\begin{frame}
        \frametitle{RSA : \textit{Vote for Pedro}}
    \framesubtitle{Résultats}
\centering
{\LARGE Signature forgée validée !}
\begin{figure}
    \centering
    \includegraphics[width=0.5\textwidth]{scriptInitCh2.png}
\end{figure}
\vspace{0.1cm}
\begin{block}{Flag obtenu}
\texttt{crypto\{y0ur\_v0t3\_i5\_my\_v0t3\}}
\end{block}
\begin{itemize}
    \item Vote accepté par le serveur
\end{itemize}
\end{frame}
    \section{Conclusion}

Dans le cadre de ce projet, nous avons mené une analyse pratique de plusieurs concepts cryptographiques en résolvant des challenges sur la plateforme CryptoHack. Nous avons mis en place une planification claire et un dépôt GitHub partagé pour organiser notre travail en binôme et documenter efficacement nos solutions.

Notre étude a débuté par les modules de la catégorie \textit{General}, où nous
avons consolidé nos compétences sur des notions fondamentales. Nous avons
d'abord traité les différentes méthodes de représentation de l'information
à travers les défis d'encodage, puis l'opération XOR, une composante
essentielle de nombreux chiffrements. L'analyse des formats de données, via
l'étude des \textit{Certificate Transparency logs}, nous a également permis
d'explorer un mécanisme de sécurité des infrastructures web modernes.

Nous avons ensuite abordé les cryptosystèmes asymétriques, en commençant par le
protocole Diffie-Hellman. Nous avons examiné ses fondements mathématiques basés
sur la théorie des groupes et la difficulté du problème du logarithme discret.
Notre analyse a également porté sur ses vulnérabilités, où nous avons mis en
œuvre une attaque de l'homme du milieu pour forcer l'utilisation de paramètres
faibles, nous permettant ainsi de compromettre la confidentialité de l'échange.

La dernière partie de notre travail s'est concentrée sur le cryptosystème RSA,
où nous avons analysé des failles liées à des erreurs d'implémentation. Nous
avons notamment démontré la possibilité de factoriser un module composé d'un
grand nombre de facteurs premiers de petite taille et d'exploiter un exposant
public faible pour forger une signature. Ces challenges ont mis en évidence
que la sécurité théorique d'un algorithme peut être compromise par une mise
en œuvre inadéquate.

En conclusion, ce projet nous a permis de passer de la théorie cryptographique
à l'application pratique, en développant des scripts pour exploiter des
vulnérabilités concrètes. Nous avons démontré que la sécurité d'un système
cryptographique ne dépend pas uniquement de la robustesse de ses fondements
mathématiques, mais également de la rigueur de son implémentation. Les
compétences techniques et méthodologiques que nous avons développées
constituent une base pour l'analyse de systèmes de sécurité plus complexes.

    % --- CORRECTION D'ERREUR ---
    % L'erreur de compilation indiquait qu'un \begin{frame} dans un des fichiers inclus
    % n'avait pas de \end{frame} correspondant. On l'ajoute ici pour clore la dernière diapositive.
\end{document}

