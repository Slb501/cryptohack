\section{Conclusion}

Dans le cadre de ce projet, nous avons mené une analyse pratique de plusieurs concepts cryptographiques en résolvant des challenges sur la plateforme CryptoHack. Nous avons mis en place une planification claire et un dépôt GitHub partagé pour organiser notre travail en binôme et documenter efficacement nos solutions.

Notre étude a débuté par les modules de la catégorie \textit{General}, où nous
avons consolidé nos compétences sur des notions fondamentales. Nous avons
d'abord traité les différentes méthodes de représentation de l'information
à travers les défis d'encodage, puis l'opération XOR, une composante
essentielle de nombreux chiffrements. L'analyse des formats de données, via
l'étude des journaux de transparence de certificats, nous a également permis
d'explorer un mécanisme de sécurité des infrastructures web modernes.

Nous avons ensuite abordé les cryptosystèmes asymétriques, en commençant par le
protocole Diffie-Hellman. Nous avons examiné ses fondements mathématiques basés
sur la théorie des groupes et la difficulté du problème du logarithme discret.
Notre analyse a également porté sur ses vulnérabilités, où nous avons mis en
œuvre une attaque de l'homme du milieu pour forcer l'utilisation de paramètres
faibles, nous permettant ainsi de compromettre la confidentialité de l'échange.

La dernière partie de notre travail s'est concentrée sur le cryptosystème RSA,
où nous avons analysé des failles liées à des erreurs d'implémentation. Nous
avons notamment démontré la possibilité de factoriser un module composé d'un
grand nombre de facteurs premiers de petite taille et d'exploiter un exposant
public faible pour forger une signature. Ces challenges ont mis en évidence
que la sécurité théorique d'un algorithme peut être compromise par une mise
en œuvre inadéquate.

En conclusion, ce projet nous a permis de passer de la théorie cryptographique
à l'application pratique, en développant des scripts pour exploiter des
vulnérabilités concrètes. Nous avons démontré que la sécurité d'un système
cryptographique ne dépend pas uniquement de la robustesse de ses fondements
mathématiques, mais également de la rigueur de son implémentation. Les
compétences techniques et méthodologiques que nous avons développées
constituent une base pour l'analyse de systèmes de sécurité plus complexes.