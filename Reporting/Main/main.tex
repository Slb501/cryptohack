% --- FICHIER PRINCIPAL : main.tex ---

\documentclass[12pt, a4paper]{article}

% On importe toute la configuration (packages, commandes, etc.)
% depuis un fichier dédié pour plus de clarté.
% Pour gérer l'encodage des caractères, indispensable pour les accents
\usepackage[utf8]{inputenc}
% Pour la gestion de la langue française (césure, typographie, etc.)
\usepackage[french]{babel}
% Pour inclure des images
\usepackage{graphicx}
% Pour forcer le placement des figures avec [H]
\usepackage{float}
% Pour gérer les marges du document
\usepackage[top=2.5cm, bottom=2.5cm, left=2.5cm, right=2.5cm]{geometry}

\usepackage{amsmath}

% Tableaux flexibles et césure dans contenus longs
\usepackage{tabularx}
\usepackage{array}
\usepackage{seqsplit}

\usepackage{titling}

% Pour avoir des liens cliquables dans le PDF (table des matières, références)
\usepackage{hyperref}
\hypersetup{
    colorlinks=true,
    linkcolor=blue,
    filecolor=magenta,
    urlcolor=cyan,
}

% --- INFORMATIONS DU DOCUMENT ---
% Ces informations seront utilisées dans la page de garde ci-dessous
\title{Rapport de Projet : CryptoHack}
\author{Victor Bailleul \and Sébastien Leglise}
\date{\today}

% --- DÉBUT DU DOCUMENT ---
\begin{document}

% --- PAGE DE GARDE PERSONNALISÉE ---
\begin{titlepage}
    \centering % Centre tout le contenu de la page de garde

    % Espace pour le logo principal (CryptoHack)
    \vspace*{1cm} % Espace en haut de la page
    \begin{figure}[h!]
        \centering
        % --- AJOUTEZ LE LOGO CRYPTOHACK ICI ---
        % Décommentez la ligne suivante et remplacez "logo_cryptohack.png"
        \includegraphics[width=0.5\textwidth]{Images/Others/logo_cryptohack.png}
            \end{figure}
    \vfill % Pousse le titre vers le centre de la page

    % Affiche le titre en gros caractères
    {\Huge\bfseries \thetitle\par}
    \vspace{1cm} % Espace vertical

    % --- AJOUT DU TITRE DE LA MATIÈRE ---
    {\Large\itshape UE Communication Scientifique\par}
    \vspace{1cm}

    % Affiche les auteurs
    {\Large \theauthor\par}
    \vspace{1cm} % Espace vertical

    % Affiche la date
    {\large \today\par}

    \vspace{2cm}
    % Nom de l'université et de l'école
    {\large Université de Caen Normandie \\ Ecole Nationale Supérieure d'Ingénieurs de Caen \\ Année 2023-2024\par}
    % Espace réservé pour les logos de l'université et de l'école
    \begin{figure}[h!]
        \centering
        % Minipage pour le premier logo (à gauche)
        \begin{minipage}{0.35\textwidth}
            \centering
             \includegraphics[width=\linewidth]{Images/Others/logo_unicaen.png}
        \end{minipage}
        \hfill % Ajoute un espace horizontal flexible entre les logos
        % Minipage pour le second logo (à droite)
        \begin{minipage}{0.35\textwidth}
            \centering
            \includegraphics[width=\linewidth]{Images/Others/logo_ensicaen.png}
        \end{minipage}
    \end{figure}
    


\end{titlepage}
% --- FIN DE LA PAGE DE GARDE ---
% Page blanche après la page de garde
\clearpage
\thispagestyle{empty}
\null
\clearpage

% --- PARTIE PRÉLIMINAIRE ---
    % On s'assure d'être sur une nouvelle page
    \clearpage

    % On insère la table des matières
    \tableofcontents
    
    % On force CETTE page (la première du sommaire) à ne pas avoir de numéro
    \thispagestyle{empty} 

    % Si le sommaire fait plus d'une page, on s'assure que les suivantes
    % n'ont pas de numéro non plus.
    \pagestyle{empty}

% --- CORPS DU DOCUMENT ---
    \clearpage 
    \pagestyle{fancy}
    % La commande suivante réinitialise le style de page et la numérotation
    \pagenumbering{arabic}
% --- CORPS DU DOCUMENT ---
    \clearpage
    \pagenumbering{arabic} % On passe en chiffres arabes ET on remet le compteur à 1

% --- INCLUSION DES CHAPITRES ---
    % --- SLIDES : Introduction ---

\section{Introduction}

\begin{frame}{Contexte du projet}
    \begin{itemize}
        \item Présentation de la plateforme CryptoHack.
        \item Rôle de la cryptographie en cybersécurité.
        \item Intérêt des CTF pour l'apprentissage.
    \end{itemize}
\end{frame}
    \section{Gestion de projet}
\subsection{Répartition du travail}
Avant de débuter notre travail sur les différents challenges de
\textit{CryptoHack}, nous avons pris le temps de planifier notre
organisation, notamment en tenant compte de notre emploi du temps déjà
chargé (plusieurs cours et contrôles continus à gérer en parallèle). Nous
avons donc fixé une \textbf{date de début précise}, afin de disposer d’un
cadre temporel clair et d’assurer une répartition équilibrée de la charge
de travail.

Une fois cette date établie, nous avons défini des \textbf{objectifs
initiaux} afin d’avoir une vision claire de notre progression. Dans un
premier temps, nous avons convenu de nous concentrer entièrement sur le
module \textit{General}, dans le but de nous familiariser avec la
plateforme \textit{CryptoHack} et de comprendre le format des exercices
proposés. Cette phase de découverte avait également pour but d’identifier
les outils nécessaires à la résolution des énigmes (Python, bibliothèques
de cryptographie, etc.) et d’évaluer le niveau de difficulté des différents
types de challenges.

Pour structurer notre travail, nous avons mis en place une
\textbf{répartition des tâches} claire. L’un de nous s’est concentré sur
les modules \textit{Encoding} et \textit{XOR}, tandis que l’autre s’est
chargé des parties \textit{Mathematics} et \textit{Data Formats}. Cette
division nous a permis de progresser en parallèle tout en couvrant un
large spectre de thématiques cryptographiques. Nous avons également fixé
une première \textbf{deadline commune} pour la fin de cette phase, afin de
pouvoir ensuite faire un point global sur l’avancement du projet.

\subsection{Gestion d'un dépôt GitHub}
Dès le début du projet, nous avons créé un \textbf{dépôt GitHub partagé},
servant de point central pour le suivi et la gestion de notre travail. Ce
dépôt nous a permis d’organiser nos scripts de résolution, nos notes et les
différents fichiers associés aux challenges. L’utilisation de GitHub s’est
révélée particulièrement utile pour la \textbf{collaboration asynchrone},
notamment lorsque nos disponibilités ne coïncidaient pas.

Chaque membre disposait d’un accès complet au dépôt et pouvait le mettre à
jour dès qu’il estimait qu’une contribution était suffisamment stable ou
pertinente. Les commits étaient accompagnés de messages explicites
décrivant les modifications apportées, ce qui facilitait la compréhension
de l’évolution du projet.

\subsection{Méthodologie GitHub}
Afin d’assurer une organisation cohérente et efficace, nous avons adopté une
\textbf{méthodologie de travail simple mais structurée} basée sur les
bonnes pratiques Git.

Chaque fois qu’un challenge était résolu, le code correspondant était
ajouté dans un dossier thématique (par exemple, \textit{Encoding/},
\textit{XOR/}, \textit{Mathematics/}, etc.).

Nous avons également convenu d’un rythme de mise à jour régulier du dépôt :
chacun pouvait pousser ses modifications après vérification, en veillant à
ne pas écraser le travail de l’autre. Lorsque cela s’avérait nécessaire,
nous communiquions directement pour fusionner ou réorganiser certaines
branches, garantissant ainsi une cohérence globale dans la structure du
projet.

Une fois les premiers modules terminés, nous avons commencé la rédaction du
rapport sur \LaTeX{}. Là encore, nous avons établi une \textbf{structure
commune} afin d’uniformiser notre style et notre présentation. Nous avons
fixé une nouvelle deadline pour la finalisation de cette partie, ce qui
nous a permis de maintenir un rythme constant.

Par la suite, nous avons décidé d’ajouter deux modules supplémentaires à nos
objectifs : \textit{Diffie-Hellman} et \textit{RSA}. Pour cette nouvelle
étape, chacun s’est vu attribuer un module et a évalué, en fonction de son
temps disponible, quels challenges étaient les plus pertinents à traiter.
Cette approche flexible mais organisée nous a permis de continuer à avancer
efficacement, tout en tenant compte de nos contraintes personnelles et
académiques.

---
En somme, cette gestion de projet s’est appuyée sur une
\textbf{communication régulière}, une \textbf{planification claire} et une
\textbf{utilisation rigoureuse de GitHub}. Cela nous a permis de progresser
de manière structurée, d’assurer la cohérence de nos contributions et de
produire un travail collectif de qualité.
    \section{Challenges : General}


\subsection{Encodage (\textit{Encoding})}

Cette première sous-partie de la catégorie \textit{General} aborde les différentes
méthodes de représentation de l'information, essentielles au transport
et à l'échange de données. La maîtrise des conversions entre des formats
comme le binaire, l'hexadécimal ou le \textit{Base64} constitue un prérequis
indispensable pour aborder des défis cryptographiques plus complexes. Il
est fondamental de bien distinguer l'encodage du chiffrement : le premier
est une transformation de format, publique et réversible, qui ne vise pas
à garantir la confidentialité, contrairement au second.

Nous avons décidé de présenter le dernier challenge de cette partie,
nommé \textit{Encoding challenge}.

\subsubsection{Objectifs}
L'objectif de ce challenge consiste à développer un script pour
automatiser l'interaction avec un serveur distant de \textit{CryptoHack}. Le
processus implique la réception de données encodées selon diverses
méthodes (\textit{Base64}, hexadécimal, \textit{ROT13}, \textit{BigInt}, et UTF-8), leur décodage
approprié, puis le renvoi de la valeur décodée au serveur.

Pour valider le challenge et obtenir le flag, il est impératif d'exécuter
cette séquence de réception, décodage et renvoi avec succès cent fois
consécutives. Cette contrainte requiert une solution
automatisée, capable de gérer dynamiquement les différents types
d'encodage rencontrés.

\subsubsection{Méthode}
Le challenge met à disposition un script partiel qui présente la manière
d'envoyer et recevoir des données avec le serveur, ainsi que le script
exécuté côté serveur (cf. \hyperref[annexe:script-server-encoding]{Annexe A}) pour vérifier les valeurs qui lui ont été
transmises. Ces scripts nous ont permis de comprendre le format des
données transmises.

La communication avec le serveur s'effectue via l'échange d'objets au
format JSON. Pour chaque itération du challenge, le serveur envoie une
requête structurée de la manière suivante~:

\begin{verbatim}
{
    "type": "type_d_encodage",
    "encoded": "donnees_encodees"
}
\end{verbatim}

Notre script doit alors analyser cette requête, appliquer la méthode de
décodage appropriée, et renvoyer la solution au serveur sous le format
JSON attendu~:

\begin{verbatim}
{
    "decoded": "donnees_decodees"
}
\end{verbatim}

Le script côté serveur vérifie alors la validité des données décodées,
puis si cela est valide, crée un nouveau challenge. Si notre script
résout cent challenges, la prochaine requête au serveur permettra
d'afficher le \textit{flag} dans la sortie standard.

\subsubsection{Résultat}

\begin{figure}[H]
    \centering
    % La commande pour insérer l'image.
    % 'width=0.8\linewidth' signifie que l'image fera 80% de la largeur du
    % texte.
    \includegraphics[width=0.8\linewidth]{Images/Encode/encode_chall_result.png}

    % La légende qui apparaîtra sous l'image.
    \caption{Capture d'écran illustrant l'obtention du \textit{flag} après le décodage 
    automatique de cent chaînes de caractères envoyées par le serveur de \textit{CryptoHack}.}

    % L'étiquette pour y faire référence plus tard.
    \label{fig:encodeChall}
\end{figure}

Le script présenté en \hyperref[annexe:script-res-encoding]{Annexe B} du rapport a permis la bonne résolution du challenge
(cf. \hyperref[fig:encodeChall]{Figure 1}) et l'obtention du \textit{flag} suivant :

\begin{center}
    \texttt{crypto\{3nc0d3\ d3c0d3\ 3nc0d3\}}
\end{center}

\subsection{XOR}

La deuxième sous-partie de la catégorie \textit{General} est consacrée à
l'opération XOR (\textit{ou} exclusif), un concept fondamental en cryptographie
constituant l'une des briques de base de nombreux algorithmes de
chiffrement. Sa simplicité de mise en œuvre et ses propriétés
mathématiques uniques en font un outil puissant pour manipuler
l'information.

Comprendre le fonctionnement du XOR est une étape cruciale, car il se
situe à la frontière entre l'opération logique et le chiffrement. L'une
de ses propriétés essentielles est sa réversibilité: appliquer deux fois
la même clé XOR à une donnée permet de retrouver la donnée originale
($A \oplus K \oplus K = A$). Cette caractéristique est au cœur de son
utilisation dans des chiffrements à flux comme le \textit{One-Time Pad}.

Nous avons décidé de présenter le challenge le plus représentatif de
cette partie, nommé \textit{Lemur XOR}.

\subsubsection{Objectifs}
Le but de ce challenge est de retrouver le \textit{flag}, à partir de
deux images fournies : \texttt{lemur.png} et \texttt{flag.png} (cf. \hyperref[fig:lemurChall]{Figure 2}). L'énoncé
nous apprend que ces deux images ont été chiffrées avec l'opération XOR en
utilisant la même clé secrète.

\begin{figure}[htbp] % L'environnement figure global
    \centering % Pour centrer le bloc des deux images

    \begin{minipage}{0.48\textwidth}
        \centering
        \includegraphics[width=\linewidth]{Images/Lemur/flag.png}
        % Première image
    \end{minipage}
    \hfill % Ajoute un espace horizontal flexible entre les images
    \begin{minipage}{0.48\textwidth}
        \centering
        \includegraphics[width=\linewidth]{Images/Lemur/lemur.png}
        % Deuxième image
    \end{minipage}

    \caption{Les deux images chiffrés avec une clé secrète commune. Sur la gauche \texttt{flag.png}, sur la droite \texttt{lemur.png}.}
    \label{fig:lemurChall}
\end{figure}

Le principe de résolution repose sur le fait qu'appliquer deux fois un XOR
avec la même clé annule l'opération. En effectuant un XOR entre les deux
images chiffrées que nous possédons, la clé secrète commune s'élimine,
ne laissant que le résultat du XOR entre les deux images originales.
C'est sur cette image combinée que le \textit{flag} devrait devenir
visible.

\subsubsection{Méthode}
Pour résoudre ce challenge, nous avons utilisé un script en Python avec la
bibliothèque de manipulation d'images \texttt{Pillow}. Nous avons commencé
par charger les deux images, \texttt{lemur.png} et \texttt{flag.png}.
Conformément aux instructions, nous avons appliqué l'opération XOR pixel
par pixel sur les valeurs de couleur RVB.

Nous avons donc parcouru les deux images simultanément et calculé la
nouvelle valeur de chaque pixel en effectuant un XOR sur ses composantes
rouge, verte et bleue. Nous avons ensuite utilisé ces nouveaux pixels
pour construire une image de sortie de mêmes dimensions que nous avons
sauvegardée.

\subsubsection{Résultat}

\begin{figure}[H]
    \centering
    % La commande pour insérer l'image.
    % 'width=0.8\linewidth' signifie que l'image fera 80% de la largeur du
    % texte.
    \includegraphics[width=0.8\linewidth]{Images/Lemur/xored_result.png}

    % La légende qui apparaîtra sous l'image.
    \caption{Ceci est le schéma explicatif de notre méthode.}

    % L'étiquette pour y faire référence plus tard.
    \label{fig:lemurChallRes}
\end{figure}

Le script présenté en \hyperref[annexe:script-lemur]{Annexe C} du rapport a permis la résolution du challenge en obtenant
 une image (cf. \hyperref[fig:lemurChallRes]{Figure 3}) affichant le \textit{flag} suivant :

\begin{center}
    \texttt{crypto\{X0Rly\_n0t!\}}
\end{center}


\subsection{Data formats}

Pour présenter cette gatégorie, nous avons choisi le challenge \textit{Transparency}. Pour le résoudre, nous nous appuyons sur le principe de \emph{Certificate Transparency} (CT), une mesure de sécurité imposée aux Autorités de Certification (CA) pour garantir la transparence dans la délivrance des certificats TLS.

Un certificat TLS (souvent appelé certificat SSL ou certificat numérique) remplit deux fonctions principales : il authentifie l'identité d'un site web ou d'un domaine auprès des clients (navigateurs, applications) et il permet d'établir une connexion chiffrée (TLS) entre le client et le serveur, garantissant la confidentialité et l'intégrité des données échangées.

Les \emph{CT logs} sont des bases de données publiques de type \textit{append-only} (où l'on ne peut qu'ajouter des entrées) dans lesquelles les certificats émis par les CA sont enregistrés. Aujourd'hui, les principales CA publient chaque certificat dans au moins deux logs CT publics pour qu'il soit accepté par les navigateurs modernes. Nous exploitons ces journaux pour retrouver des certificats correspondant à une clé publique donnée.

Ces logs étant audités et surveillés, nous pouvons vérifier les entrées, détecter d'éventuels certificats inattendus et contrôler la cohérence de la structure  afin de s'assurer qu'aucune entrée n'est dissimulée.

\subsubsection{Objectifs}
Nos objectifs sont doubles. D'abord, nous voulons retrouver le sous-domaine de cryptohack.org qui utilise la même clé publique que celle fournie dans le fichier \texttt{transparency.pem} au sein de son certificat TLS. Ensuite, en visitant ce sous-domaine, nous souhaitons obtenir le flag.

À travers ce challenge, nous visons d'abord à comprendre le fonctionnement d’un certificat TLS et sa structure (clé publique, signature, chaîne de confiance, etc.), ensuite de découvrir le système des \textit{Certificate Transparency logs}, puis d'apprendre à faire correspondre une clé publique à un certificat et enfin d'utiliser des outils d’investigation SSL/TLS et de recherche de certificats.

\subsubsection{Méthode}
Nous partons d'un fichier au format PEM (\textit{Privacy-Enhanced Mail}), un format standard pour les clés cryptographiques qui utilise l'encodage \textit{Base64}. La clé publique fournie est explicitée \hyperref[fig:publicKey]{Figure 4.}

\begin{figure}[H]
    \centering
\begin{lstlisting}[numbers=none]
    -----BEGIN PUBLIC KEY-----
MIIBIjANBgkqhkiG9w0BAQEFAAOCAQ8AMIIBCgKCAQEAuYj06m5q4M8SsEQwKX+5
NPs2lyB2k7geZw4rP68eUZmqODeqxDjv5mlLY2nz/RJsPdks4J+y5t96KAyo3S5g
mDqEOMG7JgoJ9KU+4HPQFzP9C8Gy+hisChdo9eF6UeWGTioazFDIdRUK+gZm81c1
iPEhOBIYu3Cau32LRtv+L9vzqre0Ollf7oeHqcbcMBIKL6MpsJMG+neJPnICI36B
ZZEMu6v6f8zIKuB7VUHAbDdQ6tsBzLpXz7XPBUeKPa1Fk8d22EI99peHwWt0RuJP
0QsJnsa4oj6C6lE+c5+vVHa6jVsZkpl2PuXZ05a69xORZ4oq+nwzK8O/St1hbNBX
sQIDAQAB
-----END PUBLIC KEY-----
\end{lstlisting}
\caption{Clé publique au format PEM fournie pour le challenge. Elle constitue le point de départ de notre recherche du certificat associé.}
    % L'étiquette pour y faire référence plus tard.
    \label{fig:publicKey}
\end{figure}

Notre méthode s'est déroulée en deux temps. Dans un premier temps, nous avons
analysé la problématique afin de comprendre comment relier l'empreinte SHA-256
d'une clé publique à un certificat TLS complet. Nous avons établi qu'un
certificat X.509 contient une clé publique et que sa représentation binaire
(DER) sert de base au calcul de l'empreinte utilisée par des services comme
\texttt{crt.sh}.

Forts de cette analyse, nous avons ensuite conçu un script Python dont les
objectifs étaient les suivants~: générer l'empreinte SHA-256 de la clé
publique fournie, interroger les journaux de transparence des certificats pour
retrouver les certificats correspondants, télécharger le certificat identifié,
en extraire le nom de domaine, et enfin y accéder pour récupérer le flag.
\subsubsection{Résultats}


Le script décrit en \hyperref[annexe:script-transparency]{Annexe D} nous a permis d'obtenir le sous-domaine recherché et d'identifier
le \textit{flag} (cf. \hyperref[tab:resultats-transparency]{Table 1}) permettant la résolution du challenge.

\begin{table}[H]
    \centering
    \renewcommand{\arraystretch}{1.5} % Augmente l'espacement vertical pour la lisibilité
    
    % On utilise tabularx pour que le tableau prenne toute la largeur du texte.
    % La première colonne est de largeur fixe (35%), la seconde (X) prend le reste.
    \begin{tabularx}{\linewidth}{| >{\bfseries}p{0.35\linewidth} | X |}
        \hline
        Empreinte (SHA-256) & \texttt{\seqsplit{29ab37df0a4e4d252f0cf12ad854bede59038fdd9cd652cbc5c222edd26d77d2}} \\ 
        \hline
        Sous-domaine identifié & \texttt{\seqsplit{thetransparencyflagishere.cryptohack.org}} \\ 
        \hline
        Flag obtenu & \texttt{crypto\{thx\_redpwn\_for\_inspiration\}} \\ 
        \hline
    \end{tabularx}
    
    \caption{Résultats obtenus pour le challenge Transparency.}
    \label{tab:resultats-transparency}
\end{table}
        \section{Challenges : Diffie-Hellman}
    \subsection{Introduction à l'échange de clés Diffie-Hellman}
    Cette catégorie aborde le protocole d'échange de clés de
    Diffie-Hellman, un mécanisme de cryptographie asymétrique.
    Proposé en 1976, il a pour objectif de résoudre le problème de la
    distribution de clés sur un canal de communication non sécurisé.

    Le principe de ce protocole repose sur l'utilisation de fonctions
    mathématiques à sens unique, dont le calcul est aisé dans une direction
    mais calculatoirement difficile à inverser. Spécifiquement, Diffie-Hellman
    s'appuie sur la difficulté du problème du logarithme discret dans
    un groupe fini. Ce procédé permet à deux interlocuteurs d'établir une clé
    secrète partagée sans transmission préalable de celle-ci, y compris en
    présence d'un adversaire observant la communication.

    Les challenges de cette section visent à analyser les fondements
    mathématiques de ce protocole, ainsi que les vulnérabilités pouvant
    résulter d'une implémentation incorrecte ou d'un choix de paramètres
    inadéquat.

    \subsection{L'attaque de l'homme du milieu (\textit{Man-in-the-Middle})}
    Cette sous-partie examine une vulnérabilité du protocole Diffie-Hellman~:
    l'attaque de l'homme du milieu (\textit{Man-in-the-Middle} ou MitM). Le protocole de
    base, dans sa forme originelle, ne fournit aucun mécanisme d'authentification
    des interlocuteurs. Il permet de s'assurer que la clé partagée est secrète,
    mais pas de vérifier l'identité de la personne avec qui on la partage.

    Une attaque MitM exploite cette absence d'authentification. Un adversaire
    se positionne entre les deux communicants, intercepte leurs messages publics
    et établit une session Diffie-Hellman distincte avec chacun d'eux. Chaque
    interlocuteur génère alors une clé secrète partagée avec l'attaquant, tout
    en croyant communiquer directement l'un avec l'autre.

    L'adversaire peut alors déchiffrer les messages, les lire, les modifier, puis
    les rechiffrer avec la clé de l'autre session avant de les transmettre au
    destinataire final. Les challenges de cette section illustrent comment
    cette interception est mise en œuvre et comment elle compromet la
    confidentialité et l'intégrité de l'échange.

    Pour illustrer ce type d'attaque, nous présentons le challenge \textit{Export-Grade}.

    \subsubsection{Objectifs}
    L'objectif de ce challenge est de compromettre la confidentialité d'un
    échange sécurisé par le protocole Diffie-Hellman. Le scénario repose sur
    une attaque de l'homme du milieu (\textit{Man-in-the-Middle}) durant la phase de
    négociation des paramètres cryptographiques.

    La vulnérabilité exploitée est la capacité pour un attaquant d'influencer
    le choix des paramètres de groupe utilisés par les deux correspondants. En
    les contraignant à utiliser un groupe de petite taille (64 bits), le
    problème mathématique du logarithme discret, sur lequel repose la sécurité
    du protocole, devient calculatoirement soluble. La résolution de ce
    problème permet de retrouver une clé privée, et par conséquent la clé
    secrète partagée utilisée pour chiffrer la communication.

    \subsubsection{Méthode}
    Notre attaque s'est déroulée en plusieurs étapes. Premièrement, nous avons
    activement intercepté la communication initiale d'Alice, qui proposait une
    liste de groupes cryptographiques. Nous avons altéré ce message en ne
    conservant que le groupe le plus faible, caractérisé par un module de 64
    bits, avant de le transmettre à Bob. L'acceptation de cette unique option
    par Bob a été relayée à Alice, établissant ainsi un accord sur un canal de
    communication affaibli.

        \begin{figure}[H]
            \centering
            % La commande pour insérer l'image.
            % 'width=0.8\linewidth' signifie que l'image fera 80% de la largeur du
            % texte.
            \includegraphics[width=0.95\linewidth]{Images/DiffieHellman/capture_mitm.png}

            % La légende qui apparaîtra sous l'image.
            \caption{Capture d'écran de la communication interceptée entre Alice et Bob en tant que \textit{Man-In-The-Middle}.}

            % L'étiquette pour y faire référence plus tard.
            \label{fig:mitmChallenge}
        \end{figure}

    Une fois cet accord forcé, notre rôle est devenu passif. Nous avons
    collecté les paramètres publics de l'échange : le module \textit{p}, le
    générateur \textit{g}, et les clés publiques d'Alice (\textit{A}) et de
    Bob (\textit{B}). La faible taille du module \textit{p} a alors permis de
    résoudre le problème du logarithme discret. L'échange réalisé avec Alice et Bob est illustré \hyperref[fig:mitmChallenge]{Figure 5}. 
    
    À l'aide d'un script Python, nous
    avons calculé la clé privée \textit{a} d'Alice à partir des valeurs publiques
    \textit{p}, \textit{g} et \textit{A}. La connaissance de cette clé privée nous a permis de reconstituer la clé
    secrète partagée ($s = B^a \pmod{p}$). Cette dernière a servi à dériver la
    clé de session AES, avec laquelle nous avons déchiffré le message final
    pour obtenir le \textit{flag}.

    \subsubsection{Résultat}
    L'attaque par manipulation des paramètres a réussi. En forçant l'usage d'un
    groupe faible, nous avons pu calculer la clé privée, reconstituer la clé de
    session et déchiffrer le message, révélant le \textit{flag} suivant :

    \begin{center}
        \texttt{crypto\{d0wn6r4d35\_4r3\_d4n63r0u5\}}
    \end{center}

    Le script développé pour la résolution du challenge est présenté \hyperref[annexe:script-exportgrade]{Annexe E}.

    \subsection{Théorie des groupes}
    Cette sous-partie aborde les structures mathématiques qui sous-tendent de
    nombreux protocoles de cryptographie asymétrique~: les groupes. En
    algèbre abstraite, un groupe est un ensemble d'éléments muni d'une
    opération binaire qui satisfait à des axiomes spécifiques (fermeture,
    associativité, existence d'un élément neutre et d'un inverse pour chaque
    élément).

    La sécurité de protocoles comme Diffie-Hellman ne repose pas sur les
    nombres en tant que tels, mais sur les propriétés structurelles de ces
    groupes mathématiques. Des concepts comme l'ordre d'un groupe, l'ordre
    d'un élément, et la notion de générateur d'un groupe
    cyclique sont des composantes directes de l'implémentation et de
    l'analyse de sécurité de ces systèmes.

    Les challenges de cette section ont pour objectif d'étudier ces
    propriétés. Ils illustrent comment les caractéristiques d'un groupe, ou le
    choix de ses paramètres, peuvent influencer la robustesse d'un schéma
    cryptographique.

    Pour illustrer cette section, nous présentons le challenge \textit{Additive}.

    \subsubsection{Objectifs}
    \begin{sloppypar}
    L'objectif de ce challenge est de calculer une clé secrète partagée dans une
    implémentation du protocole Diffie-Hellman utilisant un groupe additif.
    Contrairement à l'implémentation classique qui utilise un groupe
    multiplicatif d'entiers modulo un nombre premier, ce challenge transpose le
    problème dans une structure où l'opération de groupe est l'addition.
    \end{sloppypar}

    Le principe de sécurité reste fondé sur la difficulté d'inverser une
    fonction à sens unique. Ici, l'opération équivalente à l'exponentiation est
    la multiplication scalaire. Le but est de résoudre l'équivalent du problème
    du logarithme discret dans ce contexte additif afin de retrouver une clé
    privée, puis de reconstituer la clé secrète partagée, qui constitue le
    \textit{flag}.

    \subsubsection{Méthode}
    Le challenge nous fournit les paramètres publics d'un échange
    Diffie-Hellman additif : un module premier \textit{p}, un générateur
    \textit{g}, ainsi que les clés publiques d'Alice (\textit{A}) et de Bob
    (\textit{B}). La relation qui lie la clé privée \textit{a} à la clé
    publique \texttt{A} n'est plus $A = g^a \pmod{p}$, mais
    $A = a \cdot g \pmod{p}$.

    Le problème du logarithme discret se traduit ici par la résolution de
    l'équation $A \equiv a \cdot g \pmod{p}$ pour trouver l'inconnue
    \textit{a}. Cette équation est une congruence linéaire qui se résout
    efficacement en calculant l'inverse modulaire de \textit{g} modulo
    \textit{p}. Nous avons donc déterminé la clé privée d'Alice en calculant
    $a = A \cdot g^{-1} \pmod{p}$.

    Une fois la clé privée \textit{a} obtenue, nous avons pu calculer la clé
    secrète partagée en appliquant l'opération du groupe avec la clé publique
    de Bob. L'opération étant la multiplication scalaire, la clé secrète
    \textit{s} est obtenue par la formule $s = a \cdot B \pmod{p}$.

    \subsubsection{Résultat}
    L'analyse de la structure de groupe additive a permis d'identifier la
    méthode de résolution appropriée. Le calcul de l'inverse modulaire nous a
    donné accès à la clé privée, menant directement à la reconstitution de la
    clé secrète partagée. Nous avons obtenu le \textit{flag} suivant :

    \begin{center}
        \texttt{crypto\{cycl1c\_6r0up\_und3r\_4dd1710n?\}}
    \end{center}

        \section{RSA}
    \subsection{Objectifs généraux de la catégorie du challenge}
    % Contenu à rédiger ici
    \section{Conclusion}

Dans le cadre de ce projet, nous avons mené une analyse pratique de plusieurs concepts cryptographiques en résolvant des challenges sur la plateforme CryptoHack. Nous avons mis en place une planification claire et un dépôt GitHub partagé pour organiser notre travail en binôme et documenter efficacement nos solutions.

Notre étude a débuté par les modules de la catégorie \textit{General}, où nous
avons consolidé nos compétences sur des notions fondamentales. Nous avons
d'abord traité les différentes méthodes de représentation de l'information
à travers les défis d'encodage, puis l'opération XOR, une composante
essentielle de nombreux chiffrements. L'analyse des formats de données, via
l'étude des \textit{Certificate Transparency logs}, nous a également permis
d'explorer un mécanisme de sécurité des infrastructures web modernes.

Nous avons ensuite abordé les cryptosystèmes asymétriques, en commençant par le
protocole Diffie-Hellman. Nous avons examiné ses fondements mathématiques basés
sur la théorie des groupes et la difficulté du problème du logarithme discret.
Notre analyse a également porté sur ses vulnérabilités, où nous avons mis en
œuvre une attaque de l'homme du milieu pour forcer l'utilisation de paramètres
faibles, nous permettant ainsi de compromettre la confidentialité de l'échange.

La dernière partie de notre travail s'est concentrée sur le cryptosystème RSA,
où nous avons analysé des failles liées à des erreurs d'implémentation. Nous
avons notamment démontré la possibilité de factoriser un module composé d'un
grand nombre de facteurs premiers de petite taille et d'exploiter un exposant
public faible pour forger une signature. Ces challenges ont mis en évidence
que la sécurité théorique d'un algorithme peut être compromise par une mise
en œuvre inadéquate.

En conclusion, ce projet nous a permis de passer de la théorie cryptographique
à l'application pratique, en développant des scripts pour exploiter des
vulnérabilités concrètes. Nous avons démontré que la sécurité d'un système
cryptographique ne dépend pas uniquement de la robustesse de ses fondements
mathématiques, mais également de la rigueur de son implémentation. Les
compétences techniques et méthodologiques que nous avons développées
constituent une base pour l'analyse de systèmes de sécurité plus complexes.

% --- ANNEXES ---
    \clearpage
    \pagenumbering{roman}
    \appendix
    \newpage
    \section*{Annexes}
    \addcontentsline{toc}{section}{Annexes}

    % --- FICHIER : Annexes/annexes.tex ---

% La commande \appendix du main.tex va automatiquement transformer cette section*
% en "Annexe A : Script de résolution..."
\section*{A - Script exécuté côté serveur pour le challenge \textit{Encoding challenge}}
\label{annexe:script-server-encoding}

\begin{lstlisting}
from Crypto.Util.number import bytes_to_long, long_to_bytes
from utils import listener # this is cryptohack's server-side module and not part of python
import base64
import codecs
import random

FLAG = "crypto{????????????????????}"
ENCODINGS = [
    "base64",
    "hex",
    "rot13",
    "bigint",
    "utf-8",
]
with open('/usr/share/dict/words') as f:
    WORDS = [line.strip().replace("'", "") for line in f.readlines()]


class Challenge():
    def __init__(self):
        self.no_prompt = True # Immediately send data from the server without waiting for user input
        self.challenge_words = ""
        self.stage = 0

    def create_level(self):
        self.stage += 1
        self.challenge_words = "_".join(random.choices(WORDS, k=3))
        encoding = random.choice(ENCODINGS)

        if encoding == "base64":
            encoded = base64.b64encode(self.challenge_words.encode()).decode() # wow so encode
        elif encoding == "hex":
            encoded = self.challenge_words.encode().hex()
        elif encoding == "rot13":
            encoded = codecs.encode(self.challenge_words, 'rot_13')
        elif encoding == "bigint":
            encoded = hex(bytes_to_long(self.challenge_words.encode()))
        elif encoding == "utf-8":
            encoded = [ord(b) for b in self.challenge_words]

        return {"type": encoding, "encoded": encoded}

    #
    # This challenge function is called on your input, which must be JSON
    # encoded
    #
    def challenge(self, your_input):
        if self.stage == 0:
            return self.create_level()
        elif self.stage == 100:
            self.exit = True
            return {"flag": FLAG}

        if self.challenge_words == your_input["decoded"]:
            return self.create_level()

        return {"error": "Decoding fail"}


import builtins; builtins.Challenge = Challenge # hack to enable challenge to be run locally, see https://cryptohack.org/faq/#listener
listener.start_server(port=13377)
\end{lstlisting}

\newpage % Force le début de la nouvelle annexe sur une nouvelle page
\section*{B - Script de résolution du challenge \textit{Encoding challenge}}
\label{annexe:script-res-encoding}

\begin{lstlisting}
from pwn import * # pip install pwntools
from Crypto.Util.number import *
import codecs
import base64
import json

r = remote('socket.cryptohack.org', 13377, level = 'debug')

def json_recv():
    line = r.recvline()
    return json.loads(line.decode())

def json_send(hsh):
    request = json.dumps(hsh).encode()
    r.sendline(request)


for i in range(100):
    received = json_recv()

    print("Received type: ")
    print(received["type"])
    print("Received encoded value: ")
    print(received["encoded"])

    type = received["type"]
    encoded = received["encoded"]

    if type == "base64":
        decoded = base64.b64decode(encoded).decode()
    elif type == "hex":
        decoded = bytes.fromhex(encoded).decode()
    elif type == "rot13":
        decoded = codecs.decode(encoded, 'rot_13')
    elif type == "bigint":
        decoded = long_to_bytes(int(encoded, 16)).decode()
    elif type == "utf-8":
        decoded = bytes(encoded).decode()

    to_send = {
    "decoded": decoded
    }

    json_send(to_send)

json_recv()
\end{lstlisting}

\newpage % Force le début de la nouvelle annexe sur une nouvelle page
\section*{C - Script de résolution du challenge \textit{Lemur XOR}}
\label{annexe:script-lemur}

\begin{lstlisting}
from PIL import Image
import sys
from io import BytesIO

def get_rgb_bytes(png_path):
    with Image.open(png_path) as img:
        img = img.convert('RGB')
        return img.tobytes()

lemur = get_rgb_bytes("lemur_ed66878c338e662d3473f0d98eedbd0d.png")
flag = get_rgb_bytes("flag_7ae18c704272532658c10b5faad06d74.png")

xored_result = []
for i, c in enumerate(lemur):
    xored_byte = c ^ flag[i % len(flag)]
    xored_result.append(xored_byte)

with Image.open("lemur_ed66878c338e662d3473f0d98eedbd0d.png") as img:
    width, height = img.size

result_img = Image.frombytes('RGB', (width, height), bytes(xored_result))
result_img.save("xored_result.png")
\end{lstlisting}

\newpage % Force le début de la nouvelle annexe sur une nouvelle page
\section*{D - Script de résolution du challenge \textit{Transparency}}
\label{annexe:script-transparency}

\begin{lstlisting}
from Crypto.PublicKey import RSA
from OpenSSL.crypto import load_certificate, FILETYPE_PEM
import hashlib
import json
import requests

f = open("transparency_afff0345c6f99bf80eab5895458d8eab.pem")
key = RSA.import_key(f.read()).public_key()


sha256 = hashlib.sha256(key.exportKey(format="DER"))
fp = sha256.hexdigest()
print("fingerprint:", fp)

user_agent = 'Mozilla/5.0 (Windows NT 6.1; WOW64; rv:40.0) Gecko/20100101 Firefox/40.1'
url = "https://crt.sh/?spkisha256={hash}&output=json"

req = requests.get(url.format(hash=fp), headers={'User-Agent': user_agent})
content = req.content.decode('utf-8')
data = json.loads(content)
id = data[0]["id"]
download_url = "https://crt.sh/?d={id}"
req = requests.get(download_url.format(id=id), headers={'User-Agent': user_agent})
PEMcert = req.content.decode('utf-8')
    #obtenir le nom du sous domaine
cert = load_certificate(FILETYPE_PEM, PEMcert) 
CN = cert.get_subject().commonName 
print("Nom du domaine: ", CN)

flag_url = "https://" + CN
req = requests.get(flag_url, headers={'User-Agent': user_agent})
flag = req.content.decode('utf-8')
print("Flag:", flag)
\end{lstlisting}

\newpage % Force le début de la nouvelle annexe sur une nouvelle page
\section*{E - Script de résolution du challenge \textit{Export grade}}
\label{annexe:script-exportgrade}

\begin{lstlisting}
from Crypto.Cipher import AES
from Crypto.Util.Padding import unpad
import hashlib
import json
def is_pkcs7_padded(message):
    padding = message[-message[-1]:]
    return all(padding[i] == len(padding) for i in range(0, len(padding)))


def decrypt_flag(shared_secret: int, iv: str, ciphertext: str):
    # Derive AES key from shared secret
    sha1 = hashlib.sha1()
    sha1.update(str(shared_secret).encode('ascii'))
    key = sha1.digest()[:16]
    # Decrypt flag
    ciphertext = bytes.fromhex(ciphertext)
    iv = bytes.fromhex(iv)
    cipher = AES.new(key, AES.MODE_CBC, iv)
    plaintext = cipher.decrypt(ciphertext)

    if is_pkcs7_padded(plaintext):
        return unpad(plaintext, 16).decode('utf-8')
    else:
        return plaintext.decode('utf-8', errors='replace')

p_hex = "0xde26ab651b92a129"
g_hex = "0x2"
A_hex = "0x674f9bbabc48cd8d"
B_hex = "0xb0474b841afb4d6e"
iv = "42ec73835ae43797fa73803c035158fb"
encrypted_flag = "a398f814b44b69a83d2c9a0d8c35eaf65bc39b405af361a1f6ebeaec967c5809"

p = int(p_hex, 16)
g = int(g_hex, 16)
public_A = int(A_hex, 16)
public_B = int(B_hex, 16)

from sympy.ntheory.residue_ntheory import *
a = discrete_log(p, public_A, g) 
print(a)
shared_secret = pow(public_B, a, p)
print(decrypt_flag(shared_secret, iv, encrypted_flag))
\end{lstlisting}

\newpage % Force le début de la nouvelle annexe sur une nouvelle page
\section*{F - Script de résolution du challenge \textit{ManyPrime}}
\label{annexe:script-manyprime}

\begin{lstlisting}
from Crypto.Util.number import inverse, long_to_bytes
import primefac
import math

n = 580642391898843192929563856870897799...
e = 65537
ct = 32072149053462443414999372352732297...

factors = []
current = n
while current > 1:
    if primefac.isprime(current):
        factors.append(current)
        break
    else:
        p = primefac.ecm(current)
        print(f"p:{p}")
        factors.append(p)
        current //= p

product = 1
for p in factors:
    product *= p
assert n == product

phi = 1
for p in factors:
    phi *= (p - 1)

d = inverse(e, phi)
pt = pow(ct, d, n)
decrypted = long_to_bytes(pt)
print(decrypted)
\end{lstlisting}

\newpage % Force le début de la nouvelle annexe sur une nouvelle page
\section*{G - Script de résolution du challenge \textit{Vote for Pedro}}
\label{annexe:script-vote}

\begin{lstlisting}
from Crypto.Util.number import bytes_to_long, long_to_bytes
import socket
import json

host = 'socket.cryptohack.org'
port = 13375
N = 22266616657574989868109324252160663470925207690694094953...
e = 3

sock = socket.socket(socket.AF_INET, socket.SOCK_STREAM)
sock.connect((host, port))
data = sock.recv(1024)
print(data)

T = b'VOTE FOR PEDRO'
c = bytes_to_long(T)

def cube_root_2_pow(c, k_max):
    s = c % 8
    for k in range(3, k_max):
        diff = s**3 - c
        d = diff // (2**k)
        t = (-d) % 2
        s = s + t * (2**k)
    return s

s = cube_root_2_pow(c, len(T)*8 + 8)
sign_hex = long_to_bytes(s).hex()

payload = {
    "option": "vote",
    "vote": sign_hex
}

sock.send(json.dumps(payload).encode())
flag = sock.recv(1024)
print(flag)

\end{lstlisting}

\end{document}