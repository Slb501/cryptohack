% Pour gérer l'encodage des caractères, indispensable pour les accents
\usepackage[utf8]{inputenc}
% Pour la gestion de la langue française (césure, typographie, etc.)
\usepackage[french]{babel}
% Pour inclure des images
\usepackage{graphicx}
% Pour forcer le placement des figures avec [H]
\usepackage{float}
% Pour gérer les marges du document
\usepackage[top=2.5cm, bottom=2.5cm, left=2.5cm, right=2.5cm]{geometry}

\usepackage{amsmath}

% Tableaux flexibles et césure dans contenus longs
\usepackage{tabularx}
\usepackage{array}
\usepackage{seqsplit}

\usepackage{titling}
% --- Dans preambule.tex ---

\usepackage{listings}
\usepackage{xcolor}

% Définition d'un style personnalisé pour les blocs de code/clés
\lstdefinestyle{mystyle}{
    backgroundcolor=\color{black!5},   % Fond gris très clair
    commentstyle=\color{green!40!black},
    keywordstyle=\color{blue},
    stringstyle=\color{red!60!black},
    basicstyle=\ttfamily\small,        % Police monospace de petite taille
    breakatwhitespace=false,
    breaklines=true,
    captionpos=b,
    keepspaces=true,
    showspaces=false,
    showstringspaces=false,
    showtabs=false,
    tabsize=2,
    frame=single,                      % Cadre simple autour du bloc
    rulecolor=\color{black!20},      % Couleur du cadre
    numbers=left,                      % Numéros de ligne à gauche
    numberstyle=\tiny\color{gray},     % Style des numéros de ligne
}

\usepackage{booktabs} % Pour des tableaux plus esthétiques
% Applique ce style à tous les listings par défaut
\lstset{style=mystyle}
% Pour avoir des liens cliquables dans le PDF (table des matières, références)
\usepackage{hyperref}
\hypersetup{
    colorlinks=false,
    linkcolor=grey,
    filecolor=magenta,
    urlcolor=cyan,
}

% --- Dans preambule.tex ---
\usepackage{tocloft}

% Réduit l'espace avant chaque entrée de type "section"
\setlength{\cftbeforesecskip}{0.5ex}

% Réduit l'espace avant chaque entrée de type "subsection"
\setlength{\cftbeforesubsecskip}{0.3ex}

\usepackage{fancyhdr} % Charge le paquet pour les en-têtes

% Active le style de page "fancy" pour tout le document
\pagestyle{fancy}

% --- Configuration de l'en-tête ---
\fancyhf{} % Efface les configurations précédentes de l'en-tête et du pied de page
\renewcommand{\headrulewidth}{0.4pt} % Ajoute une ligne de séparation sous l'en-tête

% Contenu de l'en-tête
\lhead{Rapport de Projet : CryptoHack} % Texte à gauche de l'en-tête
\rhead{Page \thepage}                 % Numéro de page à droite

% --- Configuration du pied de page (optionnel) ---
% Laissez-le vide si vous n'en voulez pas
\lfoot{V. Bailleul \& S. Leglise}      % Texte à gauche du pied de page
\cfoot{}                              % Pas de texte au centre
\rfoot{\today}                        % La date à droite