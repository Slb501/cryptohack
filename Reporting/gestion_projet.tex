\section{Gestion de projet (Seb)}
\subsection{Répartition du travail}
Avant de débuter notre travail sur les différents challenges de
\textit{CryptoHack}, nous avons pris le temps de planifier notre
organisation, notamment en tenant compte de notre emploi du temps déjà
chargé (plusieurs cours et contrôles continus à gérer en parallèle). Nous
avons donc fixé une \textbf{date de début précise}, afin de disposer d’un
cadre temporel clair et d’assurer une répartition équilibrée de la charge
de travail.

Une fois cette date établie, nous avons défini des \textbf{objectifs
initiaux} afin d’avoir une vision claire de notre progression. Dans un
premier temps, nous avons convenu de nous concentrer entièrement sur le
module \textit{General}, dans le but de nous familiariser avec la
plateforme \textit{CryptoHack} et de comprendre le format des exercices
proposés. Cette phase de découverte avait également pour but d’identifier
les outils nécessaires à la résolution des énigmes (Python, bibliothèques
de cryptographie, etc.) et d’évaluer le niveau de difficulté des différents
types de challenges.

Pour structurer notre travail, nous avons mis en place une
\textbf{répartition des tâches} claire. L’un de nous s’est concentré sur
les modules \textit{Encoding} et \textit{XOR}, tandis que l’autre s’est
chargé des parties \textit{Mathematics} et \textit{Data Formats}. Cette
division nous a permis de progresser en parallèle tout en couvrant un
large spectre de thématiques cryptographiques. Nous avons également fixé
une première \textbf{deadline commune} pour la fin de cette phase, afin de
pouvoir ensuite faire un point global sur l’avancement du projet.

\subsection{Gestion d'un dépôt GitHub}
Dès le début du projet, nous avons créé un \textbf{dépôt GitHub partagé},
servant de point central pour le suivi et la gestion de notre travail. Ce
dépôt nous a permis d’organiser nos scripts de résolution, nos notes et les
différents fichiers associés aux challenges. L’utilisation de GitHub s’est
révélée particulièrement utile pour la \textbf{collaboration asynchrone},
notamment lorsque nos disponibilités ne coïncidaient pas.

Chaque membre disposait d’un accès complet au dépôt et pouvait le mettre à
jour dès qu’il estimait qu’une contribution était suffisamment stable ou
pertinente. Les commits étaient accompagnés de messages explicites
décrivant les modifications apportées, ce qui facilitait la compréhension
de l’évolution du projet.

\subsection{Méthodologie GitHub}
Afin d’assurer une organisation cohérente et efficace, nous avons adopté une
\textbf{méthodologie de travail simple mais structurée} basée sur les
bonnes pratiques Git.

Chaque fois qu’un challenge était résolu, le code correspondant était
ajouté dans un dossier thématique (par exemple, \textit{Encoding/},
\textit{XOR/}, \textit{Mathematics/}, etc.).

Nous avons également convenu d’un rythme de mise à jour régulier du dépôt :
chacun pouvait pousser ses modifications après vérification, en veillant à
ne pas écraser le travail de l’autre. Lorsque cela s’avérait nécessaire,
nous communiquions directement pour fusionner ou réorganiser certaines
branches, garantissant ainsi une cohérence globale dans la structure du
projet.

Une fois les premiers modules terminés, nous avons commencé la rédaction du
rapport sur \LaTeX{}. Là encore, nous avons établi une \textbf{structure
commune} afin d’uniformiser notre style et notre présentation. Nous avons
fixé une nouvelle deadline pour la finalisation de cette partie, ce qui
nous a permis de maintenir un rythme constant.

Par la suite, nous avons décidé d’ajouter deux modules supplémentaires à nos
objectifs : \textit{Diffie-Hellman} et \textit{RSA}. Pour cette nouvelle
étape, chacun s’est vu attribuer un module et a évalué, en fonction de son
temps disponible, quels challenges étaient les plus pertinents à traiter.
Cette approche flexible mais organisée nous a permis de continuer à avancer
efficacement, tout en tenant compte de nos contraintes personnelles et
académiques.

---
En somme, cette gestion de projet s’est appuyée sur une
\textbf{communication régulière}, une \textbf{planification claire} et une
\textbf{utilisation rigoureuse de GitHub}. Cela nous a permis de progresser
de manière structurée, d’assurer la cohérence de nos contributions et de
produire un travail collectif de qualité.