% Déclaration de la classe du document (article, report, book, etc.)
\documentclass[12pt, a4paper]{article}

% --- PAQUETS UTILISÉS ---

% Pour gérer l'encodage des caractères, indispensable pour les accents
\usepackage[utf8]{inputenc}

% Pour la gestion de la langue française (césure, typographie, etc.)
\usepackage[french]{babel}

% Pour gérer les marges du document
\usepackage[top=2.5cm, bottom=2.5cm, left=2.5cm, right=2.5cm]{geometry}

% Pour avoir des liens cliquables dans le PDF (table des matières, références)
\usepackage{hyperref}
\hypersetup{
    colorlinks=true,
    linkcolor=blue,
    filecolor=magenta,      
    urlcolor=cyan,
}

% --- INFORMATIONS DU DOCUMENT ---

\title{Rapport de Projet : Cryptohack}
\author{Victor \and Seb}
\date{\today} % Affiche la date de compilation

% --- DÉBUT DU DOCUMENT ---

\begin{document}

% Crée la page de titre à partir des informations ci-dessus
\maketitle

% Génère la table des matières
\tableofcontents

\newpage % Commence le contenu sur une nouvelle page

% --- CONTENU DU RAPPORT ---

\section{Introduction (Victor)}
    \subsection{Présentation de cryptohack}
    % Contenu à rédiger ici
    
    \subsection{Présentation du projet}
    % Contenu à rédiger ici

\section{Gestion de projet (Seb)}
    \subsection{Répartition du travail}
    % Contenu à rédiger ici
    
    \subsection{Gestion d'un dépôt GitHub}
    % Contenu à rédiger ici
    
    \subsection{Méthodologie GitHub}
    % Contenu à rédiger ici

\section{General (Victor pour descriptions globale)}
    \subsection{Objectifs généraux de la catégorie du challenge}
    % Contenu à rédiger ici
    
    \subsection{Encoding (Encoding challenge)}
        \subsubsection{Objectifs}
        % Contenu à rédiger ici
        
        \subsubsection{Méthode}
        % Contenu à rédiger ici
        
        \subsubsection{Résultat}
        % Contenu à rédiger ici
        
    \subsection{Xor (Lemur)}
        \subsubsection{Objectifs}
        % Contenu à rédiger ici
        
        \subsubsection{Méthode}
        % Contenu à rédiger ici
        
        \subsubsection{Résultat}
        % Contenu à rédiger ici
        
    \subsection{Data formats (Transparency)}
        \subsubsection{Objectifs}
        % Contenu à rédiger ici
        
        \subsubsection{Méthode}
        % Contenu à rédiger ici
        
        \subsubsection{Résultat}
        % Contenu à rédiger ici

\section{Diffie-Hellman}
    \subsection{Objectifs généraux de la catégorie du challenge}
    % Contenu à rédiger ici
    
    \subsection{Man in the middle (Export grade)}
    % Contenu à rédiger ici

\section{RSA}
    \subsection{Objectifs généraux de la catégorie du challenge}
    % Contenu à rédiger ici

% --- FIN DU DOCUMENT ---

\end{document}
