\section{Introduction}

\subsection{Contexte du projet}
Ce rapport présente une série de challenges de cryptographie réalisés sur la
plateforme CryptoHack. L'objectif est de documenter les approches
méthodologiques et les solutions techniques que nous avons mis en oeuvre en
binome pour résoudre les défis proposés. En préambule, il convient de situer
le contexte de ce travail en présentant d'une part le rôle fondamental de la
cryptographie en cybersécurité, et d'autre part l'intérêt des plateformes de
type \textit{Capture The Flag} (CTF) comme outil d'apprentissage.

\subsection{Place de la cryptographie en cybersécurité}
La cryptographie est une discipline scientifique qui constitue l'un des
piliers de la sécurité des systèmes d'information. Son objet est de
développer des techniques permettant de protéger l'information contre toute
modification ou accès non autorisé. En cybersécurité, la cryptographie vise
à garantir plusieurs principes de sécurité fondamentaux.

\paragraph{La confidentialité}
Ce principe assure que seules les entités autorisées puissent accéder aux
données. L'outil principal pour atteindre cet objectif est le chiffrement,
qui consiste à transformer une information (le texte clair) en une forme
inintelligible (le texte chiffré) à l'aide d'une clé secrète.

\paragraph{L'intégrité}
Ce principe garantit que les données n'ont pas été altérées ou corrompues,
que ce soit de manière accidentelle ou intentionnelle, durant leur stockage
ou leur transmission. Les fonctions de hachage et les codes
d'authentification de message (MAC) sont des mécanismes cryptographiques
courants pour vérifier l'intégrité.

\paragraph{L'authenticité}
Ce principe permet de vérifier l'identité d'une entité (un utilisateur, un
serveur, etc.). Les certificats numériques et les signatures numériques sont
des exemples de techniques cryptographiques assurant l'authentification.

\paragraph{La non-répudiation}
Ce principe empêche une entité de nier avoir effectué une action, comme
l'envoi d'un message ou la validation d'une transaction. La signature
numérique est le mécanisme de base pour fournir cette garantie.\\

De la sécurisation des communications sur Internet (protocoles TLS/SSL) à
la protection des données stockées, en passant par la sécurisation des
transactions financières et la protection de la vie privée, les
applications de la cryptographie sont omniprésentes et critiques. L'étude
de ses mécanismes et de leurs implémentations est par conséquent
essentielle pour tout praticien de la cybersécurité.

\subsection{Les plateformes de CTF dans l'apprentissage}
Les challenges de type \textit{Capture The Flag} (CTF) sont des exercices de
cybersécurité offensifs et/ou défensifs. Les participants doivent résoudre
des épreuves pour trouver une chaîne de caractères secrète, appelée
\textit{flag}, cachée dans un système, un fichier ou encore un chiffré.

Les plateformes de CTF comme \textit{root-me}, \textit{TryHackMe} ou encore
\textit{CryptoHack} sont des environnements d'apprentissage pratiques et
contrôlés où les utilisateurs peuvent appliquer des connaissances théoriques
à travers des exercices thématiques. Cette approche par la pratique favorise
une compréhension approfondie des vulnérabilités et des techniques
d'exploitation.

Ces plateformes couvrent un large éventail de domaines de la cybersécurité,
tels que l'exploitation de binaires, la rétro-ingénierie (\textit{reverse
engineering}), l'analyse forensique (\textit{digital forensics}), la
sécurité des applications web et la cryptographie.

\subsection{Présentation de CryptoHack}
CryptoHack est une plateforme en ligne dédiée à l'apprentissage de la
cryptographie moderne. Elle propose une série de challenges de type CTF qui
permettent aux utilisateurs de se familiariser avec les principes, les
algorithmes et les attaques cryptographiques. La plateforme est conçue pour
être progressive, avec des défis de difficulté croissante.

Les challenges sur CryptoHack se concentrent sur l'identification et
l'exploitation de failles dans des implémentations de protocoles et
d'algorithmes cryptographiques largement utilisés, tels que AES, RSA ou
Diffie-Hellman.

Chaque challenge est conçu pour illustrer un concept cryptographique
spécifique ou une vulnérabilité connue. Les utilisateurs sont amenés à
interagir avec des serveurs distants, à analyser du code source et à
développer leurs propres scripts, principalement en Python, pour automatiser
les attaques et récupérer les \textit{flags}.
