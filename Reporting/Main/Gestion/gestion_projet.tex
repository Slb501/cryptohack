\section{Gestion de projet}
\subsection{Organisation de la charge de travail}

Afin d'organiser notre projet, nous avons établi un calendrier de travail
s'étendant sur les trois semaines précédant la date de rendu. Cette
planification visait à définir un cadre temporel clair et à assurer une
répartition équilibrée de la charge de travail.

La première phase a été consacrée à l'exploration du module \textit{General}
de la plateforme \textit{CryptoHack}. L'objectif était de nous familiariser
avec l'environnement, d'identifier les outils techniques nécessaires et
d'évaluer la nature des challenges. Pour optimiser notre progression, nous
nous sommes réparti les sous-modules~: l'un s'est chargé des sections
\textit{Encoding} et \textit{XOR}, tandis que l'autre a traité les sections
\textit{Mathematics} et \textit{Data Formats}. Cette étape s'est achevée à
l'échéance que nous avions fixée, deux semaines avant le rendu, nous
permettant de synchroniser nos avancées.

Lors de la deuxième phase, nous avons abordé deux thématiques de la
cryptographie asymétrique~: RSA et Diffie-Hellman. En conservant une
méthodologie de travail parallèle, chaque membre du binôme s'est concentré
sur l'une de ces catégories, avec une échéance fixée à une semaine du rendu.

Enfin, la dernière semaine a été entièrement dédiée à la rédaction de ce
rapport, nous permettant de synthétiser et de documenter l'ensemble des
travaux réalisés.

\subsection{Gestion d'un dépôt GitHub}
Dès le début du projet, nous avons créé un dépôt GitHub partagé, afin de 
faciliter l'échange des productions écrites. Ce
dépôt nous a permis d’organiser nos scripts de résolution, nos notes et les
différents fichiers associés aux challenges.

Chaque membre disposait d’un accès complet au dépôt et pouvait le mettre à
jour dès qu’il estimait qu’une contribution était suffisamment stable ou
pertinente. Les \textit{commits} étaient accompagnés de messages explicites
décrivant les modifications apportées, ce qui facilitait la compréhension
de l’évolution du projet.

\subsection{Méthodologie GitHub}

Pour garantir une collaboration efficace, notre méthodologie de travail s'est
appuyée sur une structure de dépôt claire et des pratiques Git rigoureuses.

Chaque script de résolution a été classé dans un répertoire thématique
correspondant à sa catégorie (\textit{Encoding/}, \textit{XOR/}, etc.),
assurant ainsi une organisation logique des livrables. Nous avons maintenu
un rythme de mise à jour régulier, chaque membre étant responsable de
pousser ses contributions après validation.

Afin de minimiser les conflits de fusion (\textit{merge conflicts}), nous avons
limité le travail simultané sur les fichiers communs. Cette approche a été
facilitée par une communication directe pour la coordination et par des
choix structurels, comme l'adoption d'une architecture modulaire pour ce
rapport en \LaTeX{} plutôt que de travailler sur un fichier unique. Cette
méthode a garanti la cohérence et l'intégrité du projet tout au long de son
développement.