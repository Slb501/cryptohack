\section{Gestion de projet (Seb)}
\subsection{Répartition du travail}
Avant de débuter les challenges \textit{CryptoHack}, nous avons planifié notre organisation en tenant compte de notre emploi du temps académique. Nous avons fixé une \textbf{date de début précise} pour disposer d'un cadre temporel clair et assurer une répartition équilibrée du travail. Une fois cette date établie, nous avons défini des \textbf{objectifs initiaux} en commençant par le module \textit{General} pour nous familiariser avec la plateforme, comprendre le format des exercices, identifier les outils nécessaires (Python, bibliothèques de cryptographie) et évaluer le niveau de difficulté.

Pour structurer notre travail, nous avons mis en place une \textbf{répartition des tâches} claire : un membre sur les modules \textit{Encoding} et \textit{XOR}, l'autre sur \textit{Mathematics} et \textit{Data Formats}. Cette division nous a permis de progresser en parallèle tout en couvrant un large spectre thématique. Nous avons également fixé une \textbf{deadline commune} pour cette phase afin de faire un point global sur l'avancement du projet. Une fois cette deadline atteinte, nous avons entamé la rédaction du rapport tout en poursuivant avec deux nouveaux modules : \textit{RSA} pour l'un et \textit{Diffie-Hellman} pour l'autre. Une nouvelle deadline a été établie pour finaliser l'ensemble du projet, incluant la complétion des derniers challenges et la rédaction définitive du rapport.

\subsection{Gestion d'un dépôt GitHub}
Dès le début du projet, nous avons créé un \textbf{dépôt GitHub partagé},
servant de point central pour le suivi et la gestion de notre travail. Ce
dépôt nous a permis d’organiser nos scripts de résolution, nos notes et les
différents fichiers associés aux challenges. L’utilisation de GitHub s’est
révélée particulièrement utile pour la \textbf{collaboration asynchrone},
notamment lorsque nos disponibilités ne coïncidaient pas.

Chaque membre disposait d’un accès complet au dépôt et pouvait le mettre à
jour dès qu’il estimait qu’une contribution était suffisamment stable ou
pertinente. Les commits étaient accompagnés de messages explicites
décrivant les modifications apportées, ce qui facilitait la compréhension
de l’évolution du projet.

\subsection{Méthodologie GitHub}
Notre utilisation de GitHub a suivi une méthodologie rigoureuse pour garantir l'efficacité de notre collaboration. Chaque solution de challenge était systématiquement organisée dans des dossiers thématiques dédiés. 

Nous avons également convenu d’un rythme de mise à jour régulier du dépôt :
chacun pouvait pousser ses modifications après vérification, en veillant à
ne pas écraser le travail de l’autre. Lorsque cela s’avérait nécessaire,
nous communiquions directement pour fusionner ou réorganiser certaines
branches, garantissant ainsi une cohérence globale dans la structure du
projet.

Une fois les premiers modules finalisés, nous avons initié la rédaction du rapport en \LaTeX{} en adoptant une structure commune pour uniformiser la présentation.  Lorsque nous avons abordé les modules \textit{RSA} et \textit{Diffie-Hellman}, nous avons restructuré notre organisation \LaTeX{} en plusieurs dossiers thématiques pour mieux classer les sections du rapport. Cette approche nous a permis de maintenir une avancée constante jusqu'à la finalisation complète du projet, en respectant les délais que nous nous étions fixés.