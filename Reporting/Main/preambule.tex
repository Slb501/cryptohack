% --- FICHIER DE PRÉAMBULE : preambule.tex ---

% Pour gérer l'encodage des caractères, indispensable pour les accents
\usepackage[utf8]{inputenc}
% Pour la gestion de la langue française (césure, typographie, etc.)
\usepackage[french]{babel}
% Pour inclure des images
\usepackage{graphicx}
% Pour forcer le placement des figures avec [H]
\usepackage{float}
% Pour gérer les marges du document
\usepackage[top=2.5cm, bottom=2.5cm, left=2.5cm, right=2.5cm]{geometry}

% Pour avoir des liens cliquables dans le PDF (table des matières, références)
\usepackage{hyperref}
\hypersetup{
    colorlinks=true,
    linkcolor=blue,
    filecolor=magenta,
    urlcolor=cyan,
}% --- FICHIER DE PRÉAMBULE : preamble.tex ---

% Pour gérer l'encodage des caractères, indispensable pour les accents
\usepackage[utf8]{inputenc}
% Pour la gestion de la langue française (césure, typographie, etc.)
\usepackage[french]{babel}
% Pour inclure des images
\usepackage{graphicx}
% Pour forcer le placement des figures avec [H]
\usepackage{float}
% Pour gérer les marges du document
\usepackage[top=2.5cm, bottom=2.5cm, left=2.5cm, right=2.5cm]{geometry}

% Pour avoir des liens cliquables dans le PDF (table des matières, références)
\usepackage{hyperref}
\hypersetup{
    colorlinks=true,
    linkcolor=blue,
    filecolor=magenta,
    urlcolor=cyan,
}